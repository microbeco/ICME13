% Options for packages loaded elsewhere
\PassOptionsToPackage{unicode}{hyperref}
\PassOptionsToPackage{hyphens}{url}
%
\documentclass[
]{book}
\usepackage{amsmath,amssymb}
\usepackage{iftex}
\ifPDFTeX
  \usepackage[T1]{fontenc}
  \usepackage[utf8]{inputenc}
  \usepackage{textcomp} % provide euro and other symbols
\else % if luatex or xetex
  \usepackage{unicode-math} % this also loads fontspec
  \defaultfontfeatures{Scale=MatchLowercase}
  \defaultfontfeatures[\rmfamily]{Ligatures=TeX,Scale=1}
\fi
\usepackage{lmodern}
\ifPDFTeX\else
  % xetex/luatex font selection
\fi
% Use upquote if available, for straight quotes in verbatim environments
\IfFileExists{upquote.sty}{\usepackage{upquote}}{}
\IfFileExists{microtype.sty}{% use microtype if available
  \usepackage[]{microtype}
  \UseMicrotypeSet[protrusion]{basicmath} % disable protrusion for tt fonts
}{}
\makeatletter
\@ifundefined{KOMAClassName}{% if non-KOMA class
  \IfFileExists{parskip.sty}{%
    \usepackage{parskip}
  }{% else
    \setlength{\parindent}{0pt}
    \setlength{\parskip}{6pt plus 2pt minus 1pt}}
}{% if KOMA class
  \KOMAoptions{parskip=half}}
\makeatother
\usepackage{xcolor}
\usepackage{color}
\usepackage{fancyvrb}
\newcommand{\VerbBar}{|}
\newcommand{\VERB}{\Verb[commandchars=\\\{\}]}
\DefineVerbatimEnvironment{Highlighting}{Verbatim}{commandchars=\\\{\}}
% Add ',fontsize=\small' for more characters per line
\usepackage{framed}
\definecolor{shadecolor}{RGB}{248,248,248}
\newenvironment{Shaded}{\begin{snugshade}}{\end{snugshade}}
\newcommand{\AlertTok}[1]{\textcolor[rgb]{0.94,0.16,0.16}{#1}}
\newcommand{\AnnotationTok}[1]{\textcolor[rgb]{0.56,0.35,0.01}{\textbf{\textit{#1}}}}
\newcommand{\AttributeTok}[1]{\textcolor[rgb]{0.13,0.29,0.53}{#1}}
\newcommand{\BaseNTok}[1]{\textcolor[rgb]{0.00,0.00,0.81}{#1}}
\newcommand{\BuiltInTok}[1]{#1}
\newcommand{\CharTok}[1]{\textcolor[rgb]{0.31,0.60,0.02}{#1}}
\newcommand{\CommentTok}[1]{\textcolor[rgb]{0.56,0.35,0.01}{\textit{#1}}}
\newcommand{\CommentVarTok}[1]{\textcolor[rgb]{0.56,0.35,0.01}{\textbf{\textit{#1}}}}
\newcommand{\ConstantTok}[1]{\textcolor[rgb]{0.56,0.35,0.01}{#1}}
\newcommand{\ControlFlowTok}[1]{\textcolor[rgb]{0.13,0.29,0.53}{\textbf{#1}}}
\newcommand{\DataTypeTok}[1]{\textcolor[rgb]{0.13,0.29,0.53}{#1}}
\newcommand{\DecValTok}[1]{\textcolor[rgb]{0.00,0.00,0.81}{#1}}
\newcommand{\DocumentationTok}[1]{\textcolor[rgb]{0.56,0.35,0.01}{\textbf{\textit{#1}}}}
\newcommand{\ErrorTok}[1]{\textcolor[rgb]{0.64,0.00,0.00}{\textbf{#1}}}
\newcommand{\ExtensionTok}[1]{#1}
\newcommand{\FloatTok}[1]{\textcolor[rgb]{0.00,0.00,0.81}{#1}}
\newcommand{\FunctionTok}[1]{\textcolor[rgb]{0.13,0.29,0.53}{\textbf{#1}}}
\newcommand{\ImportTok}[1]{#1}
\newcommand{\InformationTok}[1]{\textcolor[rgb]{0.56,0.35,0.01}{\textbf{\textit{#1}}}}
\newcommand{\KeywordTok}[1]{\textcolor[rgb]{0.13,0.29,0.53}{\textbf{#1}}}
\newcommand{\NormalTok}[1]{#1}
\newcommand{\OperatorTok}[1]{\textcolor[rgb]{0.81,0.36,0.00}{\textbf{#1}}}
\newcommand{\OtherTok}[1]{\textcolor[rgb]{0.56,0.35,0.01}{#1}}
\newcommand{\PreprocessorTok}[1]{\textcolor[rgb]{0.56,0.35,0.01}{\textit{#1}}}
\newcommand{\RegionMarkerTok}[1]{#1}
\newcommand{\SpecialCharTok}[1]{\textcolor[rgb]{0.81,0.36,0.00}{\textbf{#1}}}
\newcommand{\SpecialStringTok}[1]{\textcolor[rgb]{0.31,0.60,0.02}{#1}}
\newcommand{\StringTok}[1]{\textcolor[rgb]{0.31,0.60,0.02}{#1}}
\newcommand{\VariableTok}[1]{\textcolor[rgb]{0.00,0.00,0.00}{#1}}
\newcommand{\VerbatimStringTok}[1]{\textcolor[rgb]{0.31,0.60,0.02}{#1}}
\newcommand{\WarningTok}[1]{\textcolor[rgb]{0.56,0.35,0.01}{\textbf{\textit{#1}}}}
\usepackage{longtable,booktabs,array}
\usepackage{calc} % for calculating minipage widths
% Correct order of tables after \paragraph or \subparagraph
\usepackage{etoolbox}
\makeatletter
\patchcmd\longtable{\par}{\if@noskipsec\mbox{}\fi\par}{}{}
\makeatother
% Allow footnotes in longtable head/foot
\IfFileExists{footnotehyper.sty}{\usepackage{footnotehyper}}{\usepackage{footnote}}
\makesavenoteenv{longtable}
\usepackage{graphicx}
\makeatletter
\def\maxwidth{\ifdim\Gin@nat@width>\linewidth\linewidth\else\Gin@nat@width\fi}
\def\maxheight{\ifdim\Gin@nat@height>\textheight\textheight\else\Gin@nat@height\fi}
\makeatother
% Scale images if necessary, so that they will not overflow the page
% margins by default, and it is still possible to overwrite the defaults
% using explicit options in \includegraphics[width, height, ...]{}
\setkeys{Gin}{width=\maxwidth,height=\maxheight,keepaspectratio}
% Set default figure placement to htbp
\makeatletter
\def\fps@figure{htbp}
\makeatother
\setlength{\emergencystretch}{3em} % prevent overfull lines
\providecommand{\tightlist}{%
  \setlength{\itemsep}{0pt}\setlength{\parskip}{0pt}}
\setcounter{secnumdepth}{5}
\usepackage{booktabs}
\ifLuaTeX
  \usepackage{selnolig}  % disable illegal ligatures
\fi
\usepackage[]{natbib}
\bibliographystyle{plainnat}
\usepackage{bookmark}
\IfFileExists{xurl.sty}{\usepackage{xurl}}{} % add URL line breaks if available
\urlstyle{same}
\hypersetup{
  pdftitle={ICME 13 - Rome Edition},
  pdfauthor={Manuela Coci; Luigi Gallucci; Davide Corso},
  hidelinks,
  pdfcreator={LaTeX via pandoc}}

\title{ICME 13 - Rome Edition}
\usepackage{etoolbox}
\makeatletter
\providecommand{\subtitle}[1]{% add subtitle to \maketitle
  \apptocmd{\@title}{\par {\large #1 \par}}{}{}
}
\makeatother
\subtitle{Crash course on Nanopore sequencing for microbial ecology}
\author{Manuela Coci \and Luigi Gallucci \and Davide Corso}
\date{Last update: 2024-03-12}

\begin{document}
\maketitle

{
\setcounter{tocdepth}{1}
\tableofcontents
}
\chapter{Introduction}\label{introduction}

MicrobEco is a non-profit scientific organization dedicated to the advancement of scientific research and know-how in the field of microbiology and ecology. MICROBECO's mission is to promote the understanding and appreciation of microbes, to disseminate knowledge, strengthen collaboration, create and provide opportunities to learn about, discuss and challenge frontier issues in microbial ecology and to support the development of new technologies and applications that can benefit society. We connect scientists and we educate future generations to know and to do not fear microbes.

\section{ICME 13 - Rome}\label{icme-13---rome}

The course ``Crash course on Nanopore sequencing for microbial ecology'' corresponds to the 13th edition of the ICME- International course in microbial ecology, which annually gathers selected students to receive theoretical and practical training on one or more techniques in microbial ecology. From 11 to 14 March 2024, 25 PhD students and early career researchers will have practical experience in sequencing using Nanopore technology (MinIon) and performing bioinformatic analysis of Nanopore sequencing data. As in the tradition of ICME, The entire course is designed to bring together beginners and experts, and to create a strong collaboration between colleagues that extends beyond the days of the course.

The course is held at the laboratory of CNR-IRSA Institute of Water Research Roma Montelibretti.

Official page: \url{https://www.microbeco.org/icme13-roma-2024/}

\chapter{Introduction to the command line}\label{introduction-to-the-command-line}

\section{Set-up a terminal}\label{set-up-a-terminal}

\textbf{MacOS/Linux:} Launch terminal on your machine.

\textbf{Windows users options:}
\textbf{Windows Subsystem for Linux (WSL)} --\textgreater{} It creates an Ubuntu terminal environment where you can code just like from a linux Ubuntu terminal. This is useful for the course as well as for practice working in bash. \href{https://ubuntu.com/wsl}{from ubuntu website} and \href{https://learn.microsoft.com/en-us/windows/wsl/install}{from the windows website}

\textbf{SSH client} --\textgreater{} Windows: {[}MobaXterm{]}(~\url{https://mobaxterm.mobatek.net/download-home-edition.html} . This is a very basic ssh client, meaning, it will allow you to connect to the server and it will serve as a terminal for the course.

If you are already using Visual Studio, it needs one ssh~\href{https://code.visualstudio.com/docs/remote/ssh}{extension}~plugin to serve as a ssh.

\textbf{Git for windows} --\textgreater{} I am not sure this can be used as a ssh but, in regards to this course, it is also useful to practice coding on the terminal.

Very last-minute resource --\textgreater{} launch \href{https://bellard.org/jslinux/vm.html?url=alpine-x86.cfg&mem=192}{this terminal emulator} in a new window.

\section{Working with the command line}\label{working-with-the-command-line}

Most of the activities of the bioinformatic section of this workshop will be done using the Unix command line (Unix shell).\\
It is therefore highly recommended to have at least a basic grasp of how to get around in the Unix shell.\\
We will now dedicate one hour or so to follow some basic to learn (or refresh) the basics of the Unix shell.

\begin{quote}
{[}!question{]} What is the UNIX SHELL? What is Bash?

\begin{quote}
{[}!todo{]} The shell is a program that enables us to send commands to the computer and receive output. It is also referred to as the terminal or command line. Some computers include a default Unix Shell program.
\end{quote}

\begin{quote}
{[}!todo{]} The most popular Unix shell is Bash, Bash is a shell and a command language.
\end{quote}

For a \textbf{Mac} computer running macOS Mojave or earlier releases, the default Unix Shell is Bash.

For a \textbf{Mac} computer running macOS Catalina or later releases, the default Unix Shell is Zsh. Your default shell is available via the Terminal program within your Utilities folder.

The default Unix Shell for \textbf{Linux} operating systems is usually Bash.
\end{quote}

\section{Playing around with basic UNIX commands}\label{playing-around-with-basic-unix-commands}

\subsection{Some notes!}\label{some-notes}

These commands:

\begin{Shaded}
\begin{Highlighting}[]
\FunctionTok{mkdir}\NormalTok{ unix\_shell}
\BuiltInTok{cd}\NormalTok{ unix\_shell}
\end{Highlighting}
\end{Shaded}

\ldots are commands you need to type in the shell.
Each line is a command.
Commands have to be typed in a single line, one at a time.
After each command, hit ``Enter'' to execute it.

Things starting with a hashtag:

\begin{Shaded}
\begin{Highlighting}[]
\CommentTok{\# This is a comment and is ignored by the shell}
\end{Highlighting}
\end{Shaded}

\ldots are comments embedded in the code to give instructions to the user.
Anything in a line starting with a \texttt{\#} is ignored by the shell.

Different commands might expect different syntaxes and different types of arguments. Some times the order matters, some times it doesn't! The best way to check how to run a command is by taking a look at its manual with the command \texttt{man} or to the --help for a shorter version of it:

\begin{Shaded}
\begin{Highlighting}[]
\FunctionTok{man}\NormalTok{ mkdir}

\CommentTok{\# You can scroll down by hitting the space bar}
\CommentTok{\# To quit, hit "q"}

\FunctionTok{mkdir} \AttributeTok{{-}h}

\CommentTok{\# did it work?}
\end{Highlighting}
\end{Shaded}

\subsection{Creating and navigating directories}\label{creating-and-navigating-directories}

First let's see where we are:

\begin{Shaded}
\begin{Highlighting}[]
\BuiltInTok{pwd}  \CommentTok{\# print working directory}
\end{Highlighting}
\end{Shaded}

Are there any files here? Let's list the contents of the folder:

\begin{Shaded}
\begin{Highlighting}[]
\FunctionTok{ls}

\CommentTok{\# or }

\ExtensionTok{ll}
\end{Highlighting}
\end{Shaded}

Let's now create a new folder called \texttt{unix\_shell}. In addition to the command (\texttt{mkdir}), we are now passing a term (also known as an argument) which, in this case, is the name of the folder we want to create:

\begin{Shaded}
\begin{Highlighting}[]
\FunctionTok{mkdir}\NormalTok{ unix\_shell}
\end{Highlighting}
\end{Shaded}

Has anything changed? How to list the contents of the folder again?

HINT (CLICK TO EXPAND)

\begin{quote}
ls
\end{quote}

\begin{center}\rule{0.5\linewidth}{0.5pt}\end{center}

And now let's enter the \texttt{unix\_shell} folder:

\begin{Shaded}
\begin{Highlighting}[]
\BuiltInTok{cd}\NormalTok{ unix\_shell}
\end{Highlighting}
\end{Shaded}

Did it work? Where are we now?

HINT

\begin{quote}
pwd
\end{quote}

\subsection{Creating a new file}\label{creating-a-new-file}

Let's create a new file called \texttt{myfile.txt} by launching the text editor \texttt{nano}:

\begin{Shaded}
\begin{Highlighting}[]
\FunctionTok{nano}\NormalTok{ myfile.txt}
\end{Highlighting}
\end{Shaded}

Now inside the nano screen:

\begin{enumerate}
\def\labelenumi{\arabic{enumi}.}
\item
  Write some text
\item
  Exit the ``writing mode'' with ctrl+x
  nano
\item
  To save the file, type \textbf{y} and hit ``Enter''
\item
  Confirm the name of the file and hit ``Enter''
\end{enumerate}

List the contents of the folder. Can you see the file we have just created?

\subsection{Copying, renaming, moving and deleting files}\label{copying-renaming-moving-and-deleting-files}

First let's create a new folder called \texttt{myfolder}. Do you remember how to do this?

HINT

\begin{quote}
mkdir myfolder
\end{quote}

\begin{center}\rule{0.5\linewidth}{0.5pt}\end{center}

And now let's make a copy of \texttt{myfile.txt}. Here, the command \texttt{cp} expects two arguments, and the order of these arguments matter. The first is the name of the file we want to copy, and the second is the name of the new file:

\begin{Shaded}
\begin{Highlighting}[]
\FunctionTok{cp}\NormalTok{ myfile.txt newfile.txt}
\end{Highlighting}
\end{Shaded}

List the contents of the folder. Do you see the new file there?

Now let's say we want to copy a file and put it inside a folder. In this case, we give the name of the folder as the second argument to \texttt{cp}:

\begin{Shaded}
\begin{Highlighting}[]
\FunctionTok{cp}\NormalTok{ myfile.txt myfolder }

\CommentTok{\# while typing myfold.. try using the TAB to predict the name of the folder!}

\FunctionTok{cp}\NormalTok{ myfile.txt myfolder/  }\CommentTok{\# it will recognise it is a directory and add the / at the end.}
\end{Highlighting}
\end{Shaded}

List the contents of \texttt{myfolder}. Is \texttt{myfile.txt} there?

\begin{Shaded}
\begin{Highlighting}[]
\FunctionTok{ls}\NormalTok{ myfolder}
\end{Highlighting}
\end{Shaded}

We can also copy the file to another folder and give it a different name, like this:

\begin{Shaded}
\begin{Highlighting}[]
\FunctionTok{cp}\NormalTok{ myfile.txt myfolder/copy\_of\_myfile.txt}
\end{Highlighting}
\end{Shaded}

List the contents of \texttt{myfolder} again. Do you see two files there?

Instead of copying, we can move files around with the command \texttt{mv}:

\begin{Shaded}
\begin{Highlighting}[]
\FunctionTok{mv}\NormalTok{ newfile.txt myfolder}
\end{Highlighting}
\end{Shaded}

Let's list the contents of the folders. Where did \texttt{newfile.txt} go?

We can also use the command \texttt{mv} to rename files:

\begin{Shaded}
\begin{Highlighting}[]
\FunctionTok{mv}\NormalTok{ myfile.txt myfile\_renamed.txt}
\end{Highlighting}
\end{Shaded}

List the contents of the folder again. What happened to \texttt{myfile.txt}?

Now, let's say we want to move things from inside \texttt{myfolder} to the current directory. Can you see what the dot (\texttt{.}) is doing in the command below? Let's try:

\begin{Shaded}
\begin{Highlighting}[]
\FunctionTok{mv}\NormalTok{ myfolder/newfile.txt .}
\end{Highlighting}
\end{Shaded}

Let's list the contents of the folders. The file \texttt{newfile.txt} was inside \texttt{myfolder} before, where is it now?

The same operation can be done in a different way. In the commands below, can you see what the two dots (\texttt{.}) are doing? Let's try:

\begin{Shaded}
\begin{Highlighting}[]
\CommentTok{\# First we go inside the folder}
\BuiltInTok{cd}\NormalTok{ myfolder}

\CommentTok{\# Then we move the file one level up}
\FunctionTok{mv}\NormalTok{ myfile.txt ..}

\CommentTok{\# And then we go back one level}
\BuiltInTok{cd}\NormalTok{ ..}
\end{Highlighting}
\end{Shaded}

Let's list the contents of the folders. The file \texttt{myfile.txt} was inside \texttt{myfolder} before, where is it now?

To remove files :

\begin{Shaded}
\begin{Highlighting}[]
\FunctionTok{rm}\NormalTok{ newfile.txt}
\end{Highlighting}
\end{Shaded}

Let's list the contents of the folder. What happened to \texttt{newfile.txt}?

And now let's delete \texttt{myfolder}:

\begin{Shaded}
\begin{Highlighting}[]
\FunctionTok{rm}\NormalTok{ myfolder}
\end{Highlighting}
\end{Shaded}

It didn't work did it? An error message came up, what does it mean?

\begin{Shaded}
\begin{Highlighting}[]
\ExtensionTok{rm:}\NormalTok{ cannot remove ‘myfolder’: Is a directory}
\end{Highlighting}
\end{Shaded}

To delete a folder we have to modify the command further by adding the flag (\texttt{-r}). Flags are used to pass additional options to the commands:

\begin{Shaded}
\begin{Highlighting}[]
\FunctionTok{rm} \AttributeTok{{-}r}\NormalTok{ myfolder}
\end{Highlighting}
\end{Shaded}

Let's list the contents of the folder. What happened to \texttt{myfolder}?

\begin{quote}
{[}!warning{]}
\textbf{In Bash, If you remove the wrong file/directory, it is gone forever!! (no recycle bin!)}
\textbf{aka BE CAREFUL!!}
\end{quote}

\chapter{16s Analysis}\label{s-analysis}

In this tutorial we are going to see/use different commands and tools to perform the 16s analysis on long reads NanoPore. However, the practical analysis will cover only some of these steps as data has been prepared in advance due to their computation and time consuming limits.

\textbf{This guide has been created with the purpose of a practical crush course and it is not intended as a complete reference but rather as a beginners pipeline to analyze NanoPore generated data.}

\section{Commands and features that we will use in our practice}\label{commands-and-features-that-we-will-use-in-our-practice}

\subsection{\texorpdfstring{\texttt{touch}}{touch}}\label{touch}

Create a new empty file.

Example: \texttt{touch\ prova.txt}

\subsection{\texorpdfstring{\texttt{echo}}{echo}}\label{echo}

The echo command is a built-in Linux feature that prints out arguments as the standard output. echo is commonly used \emph{to display text strings or command results as messages}.

Example: \texttt{echo\ "Hello\ World"}

\subsection{\texorpdfstring{\texttt{cat}}{cat}}\label{cat}

Concatenate or print the contents of a file

Example: \texttt{cat\ prova.txt}

\subsection{\texorpdfstring{\texttt{\$}}{\$}}\label{section}

Append \texttt{\$} to the variable name to access the variable value.

Example:

\begin{Shaded}
\begin{Highlighting}[]
\VariableTok{var}\OperatorTok{=}\StringTok{"Hello World"}
\BuiltInTok{echo} \VariableTok{$var}
\end{Highlighting}
\end{Shaded}

Output:

\begin{quote}
Hello World
\end{quote}

\subsection{\texorpdfstring{\texttt{chmod}}{chmod}}\label{chmod}

Change the permissions of a file or directory and make it executable.

\begin{itemize}
\tightlist
\item
  use the ``chmod +x'' command on a system file to give permission to all users to execute it.
\item
  use the ``chmod u+x'' for made the file executable for your user.
\end{itemize}

Example: \texttt{chmod\ u+x\ s01\_filtering.sh}

\subsection{\texorpdfstring{\texttt{basename}}{basename}}\label{basename}

It removes the path from a file string, providing only its filename and trailing suffix from given file names.

Example:

\begin{Shaded}
\begin{Highlighting}[]
\FunctionTok{basename}\NormalTok{ /path/to/filename.txt}
\end{Highlighting}
\end{Shaded}

Output:

\begin{quote}
filename.txt
\end{quote}

\subsection{Create a shell scripts}\label{create-a-shell-scripts}

Sometime you don't need to run a command at a time, we can pre-think, organize series of actions (a program) that you can then execute within Bash.

For example we can write a shell script that runs a series of commands and we can run the script from the terminal to execute all the steps that we have integrated in the script.

For example, lets assume that we want to visualize the first four reads from a FASTQ and redirect to a file. For this task we are going to create an empty file and write in it the following text:

\begin{Shaded}
\begin{Highlighting}[]
\CommentTok{\#!/bin/bash}

\FunctionTok{cat}\NormalTok{ /SERVER/16s\_data/mysample.FASTQ }\KeywordTok{|} \FunctionTok{head} \AttributeTok{{-}n}\NormalTok{ 4 }\OperatorTok{\textgreater{}}\NormalTok{ first\_read.fastq}

\VariableTok{var}\OperatorTok{=}\StringTok{"Hello World"}
\BuiltInTok{echo} \VariableTok{$var}
\end{Highlighting}
\end{Shaded}

Now we give the `executable' permission to our script in order to be executed:

\texttt{chmod\ u+x\ myscript.sh}

Now we execute our script:

\texttt{./myscript.sh}

\begin{itemize}
\tightlist
\item
  The first row must be \texttt{\#!/bin/bash} to allows the shell to interpret your code with bash
\item
  The symbol \texttt{\textbar{}} is the PIPE, it lets you connect actions: the output of a command is the input of the next command
\item
  The command \texttt{head} allows to print a specified number of rows from an input.
\item
  Sometime different actions cannot be linked with a PIPE, in this case we use to write the series of actions in each line, as we did in the last two rows of our scripts.
\end{itemize}

\section{Package Manager}\label{package-manager}

\subsection{Conda}\label{conda}

Conda is a powerful command line tool for package and environment management that runs on Windows, macOS, and Linux.

\url{https://conda.io/projects/conda/en/latest/user-guide/getting-started.html}

Within conda, you can create, export, list, remove, and update environments that have different versions of Python and/or packages installed in them. \emph{Switching or moving between environments is called activating the environment}. You can also share an environment file.

\subsection{Mamba (Recommended)}\label{mamba-recommended}

Mamba is a reimplementation of the conda package manager in C++, so it is faster and more convenient due to its faster dependencies solving.

\url{https://mamba.readthedocs.io/en/latest/user_guide/mamba.html}

\subsubsection{Creating a conda/mamba env}\label{creating-a-condamamba-env}

To create an environment:

\begin{Shaded}
\begin{Highlighting}[]
\ExtensionTok{conda}\NormalTok{ create }\AttributeTok{{-}n}\NormalTok{ myenv}
\end{Highlighting}
\end{Shaded}

To activate a created environment:

\begin{Shaded}
\begin{Highlighting}[]
\ExtensionTok{conda}\NormalTok{ activate myenv}
\end{Highlighting}
\end{Shaded}

For our practical analysis, conda environments have already been created.

The following environment were created and used:
16S
mamba activate /home/irsa/miniconda3/envs/ONTpp
mamba activate /home/irsa/miniconda3/envs/emu

\section{Base Calling}\label{base-calling}

Base calling is the process of translating the electronic raw signal of the sequencer into bases, i.e., ATCG and converting the raw files (FAST5) to a FASTQ files (human-readable), which contains the nucleotide sequences of the reads.

Raw data are huge in terms of storage, and since basecalling is computationally and time demanding, the fastq files are already provided.

However, to perform this step we suggest to use one of the two following tools:

\subsection{Guppy}\label{guppy}

\url{https://community.nanoporetech.com/docs/prepare/library_prep_protocols/Guppy-protocol/v/gpb_2003_v1_revax_14dec2018/guppy-software-overview}

Guppy usage:

\begin{Shaded}
\begin{Highlighting}[]
\ExtensionTok{guppy\_basecaller} \DataTypeTok{\textbackslash{}}
  \AttributeTok{{-}{-}num\_callers}\NormalTok{ 4 }\DataTypeTok{\textbackslash{}}
  \AttributeTok{{-}{-}cpu\_threads\_per\_caller}\NormalTok{ 64 }\DataTypeTok{\textbackslash{}}
  \AttributeTok{{-}{-}input\_path} \DataTypeTok{\textbackslash{}}
  \AttributeTok{{-}{-}save\_path} \DataTypeTok{\textbackslash{}}
  \AttributeTok{{-}{-}flowcell}\NormalTok{ FLO{-}MIN106 }\DataTypeTok{\textbackslash{}}
  \AttributeTok{{-}{-}kit}\NormalTok{ SQK{-}RBK004s}
\end{Highlighting}
\end{Shaded}

As we can see from the command line this tool requires the flowcell model and eventually barcoding kit information.

\subsection{Dorado}\label{dorado}

Recently released by Nanopore, Dorado is a high-performance, easy-to-use, open source basecaller for Oxford Nanopore reads, with the options of super, high and low, accuracy.

\url{https://github.com/nanoporetech/dorado}

First download the flowcell model kit (You need to know which flowcell was used):

\begin{Shaded}
\begin{Highlighting}[]
\ExtensionTok{dorado}\NormalTok{ download }\AttributeTok{{-}{-}model}\NormalTok{ dna\_r10.4.1\_e8.2\_400bps\_hac@v4.1.0}
\end{Highlighting}
\end{Shaded}

Dorado usage:

\begin{Shaded}
\begin{Highlighting}[]
\ExtensionTok{dorado}\NormalTok{ basecaller }\DataTypeTok{\textbackslash{}}
  \AttributeTok{{-}b}\NormalTok{ 36 }\DataTypeTok{\textbackslash{}}
  \AttributeTok{{-}{-}device}\NormalTok{ cpu }\DataTypeTok{\textbackslash{}}
  \AttributeTok{{-}{-}emit{-}fastq}
  \ExtensionTok{mysample.FAST5}
\end{Highlighting}
\end{Shaded}

\section{Filtering and Trimming}\label{filtering-and-trimming}

The fastq file containing the 16S sequence need to be filtered based on quality and/or read length, and optional trimmed after passing filter

\subsection{NanoFilt}\label{nanofilt}

NanoFilt - filtering and trimming of long read sequencing data

\url{https://github.com/wdecoster/nanofilt}

\emph{Requirement}

To execute this tool, you need to activate the conda environment

\texttt{mamba\ activate\ /home/irsa/miniconda3/envs/ONTpp}

\textbf{STEP TIPS:}

\begin{itemize}
\tightlist
\item
  Your input data are available in the following path: \texttt{/SERVER/16s\_data/}
\item
  Your task should be to use NanoFilt on each fastq file, using the following options:

  \begin{itemize}
  \tightlist
  \item
    \texttt{-\/-length\ LENGTH} Filter on a minimum read length
  \item
    \texttt{-\/-maxlength\ MAXLENGTH} Filter on a maximum read length
  \item
    \texttt{-\/-quality\ QUALITY} Filter on a minimum average read quality score
  \end{itemize}
\item
  Create the folder for your output files (to use in your script)
\item
  To perform this step you should create a script (Recommended name: \texttt{s01\_nanofilt.sh}), and give it permission to be executed: \texttt{chmod\ u+x\ s01\_nanofilt.sh}
\item
  Execute your script as follow: \texttt{./s01\_nanofilt.sh}
\item
  Look at the results
\end{itemize}

\textbf{{[}SPOILER{]}} - Scripts that we will use

We create an empty file called \texttt{s01\_nanofilt.sh}

\texttt{touch\ s01\_nanofilt.sh}

We can write our actions in the scripts as follows:

\begin{Shaded}
\begin{Highlighting}[]
\CommentTok{\#!/bin/bash}

\ControlFlowTok{for}\NormalTok{ i }\KeywordTok{in}\NormalTok{ /SERVER/16s\_data/}\PreprocessorTok{*}\NormalTok{fastq}
\ControlFlowTok{do}
    \VariableTok{f}\OperatorTok{=}\VariableTok{$(}\FunctionTok{basename} \StringTok{"}\VariableTok{$i}\StringTok{"}\NormalTok{ .fastq}\VariableTok{)}
    \BuiltInTok{echo} \StringTok{"}\VariableTok{$f}\StringTok{"}
    \BuiltInTok{echo} \StringTok{"}\VariableTok{$i}\StringTok{"}
    \FunctionTok{cat} \VariableTok{$i} \KeywordTok{|} \ExtensionTok{NanoFilt} \AttributeTok{{-}q}\NormalTok{ 9 }\AttributeTok{{-}l}\NormalTok{ 1200 }\AttributeTok{{-}{-}maxlength}\NormalTok{ 1800 }\OperatorTok{\textgreater{}}\NormalTok{ /home/irsa/analisi\_16s/output\_s01/}\StringTok{"}\VariableTok{$f}\StringTok{"}\NormalTok{{-}nf.fastq}
\ControlFlowTok{done}
\end{Highlighting}
\end{Shaded}

Create output directory for this script (Change \texttt{irsa} with your \texttt{utenteX} name)

\begin{verbatim}
mkdir -p /home/irsa/analisi_16s/output_s01/
\end{verbatim}

Change its permission:
\texttt{chmod\ u+x\ s01\_nanofilt.sh}

Execute it:
\texttt{./s01\_nanofilt.sh}

\section{Subsetting}\label{subsetting}

We are going to reduce the number of reads in order to use less resources for our practical analysis. \textbf{It is important to note that this step is not part of a common pipeline}.

\subsection{BBMAP Tools}\label{bbmap-tools}

BBMap - short read aligner for DNA/RNAseq, and other bioinformatic tools, including BBMap.

\url{https://github.com/BioInfoTools/BBMap}

From this step, we use a script provided by BBMAP tools collection, called: \texttt{reformat.sh}, which reformats reads to change ASCII quality encoding, interleaving, file format, or compression format.

\emph{Requirement}

To execute this tool, you need to activate the conda environment:

\texttt{mamba\ activate\ /home/irsa/miniconda3/envs/ONTpp}

\textbf{STEP TIPS:}

\begin{itemize}
\tightlist
\item
  Your input data should be available from the output folder of the previous step
\item
  Your task should be to use \texttt{reformat.sh} on \textbf{each output files obtained from the previous step}, using the following options:

  \begin{itemize}
  \tightlist
  \item
    \texttt{in=\textless{}file\textgreater{}} Input file
  \item
    \texttt{out=\textless{}outfile\textgreater{}} Ouput file
  \item
    \texttt{samplereadstarget=10000} Exact number of OUTPUT reads (or pairs) desired.
  \end{itemize}
\item
  Create the folder for your output files (to use in your script)
\item
  To perform this step you should create a script (Recommended name: \texttt{s02\_subsampling.sh}), and give it permission to be executed: \texttt{chmod\ u+x\ s02\_subsampling.sh}
\item
  Execute your script as follow: \texttt{./s02\_subsampling.sh}
\item
  Look at the results
\end{itemize}

\textbf{{[}SPOILER{]}} - Scripts that we will use

We create an empty file called \texttt{s02\_subsampling.sh}

\texttt{touch\ s02\_subsampling.sh}

We can write our actions in the scripts as follows:

\begin{Shaded}
\begin{Highlighting}[]
\CommentTok{\#!/bin/bash}

\CommentTok{\# Create output directory for this script}
\FunctionTok{mkdir} \AttributeTok{{-}p}\NormalTok{ /home/irsa/analisi\_16s/output\_s02/}

\ControlFlowTok{for}\NormalTok{ i }\KeywordTok{in}\NormalTok{ /home/irsa/analisi\_16s/output\_s01/}\PreprocessorTok{*}\NormalTok{{-}nf.fastq}
\ControlFlowTok{do}
    \VariableTok{f}\OperatorTok{=}\VariableTok{$(}\FunctionTok{basename} \StringTok{"}\VariableTok{$i}\StringTok{"} \AttributeTok{{-}nf.fastq}\VariableTok{)}
    \BuiltInTok{echo} \StringTok{"}\VariableTok{$f}\StringTok{"}
    \BuiltInTok{echo} \StringTok{"}\VariableTok{$i}\StringTok{"}
    \ExtensionTok{reformat.sh}\NormalTok{ in=}\StringTok{"}\VariableTok{$i}\StringTok{"}\NormalTok{ out=/home/irsa/analisi\_16s/output\_s02/}\StringTok{"}\VariableTok{$f}\StringTok{"}\NormalTok{{-}ss{-}nf.fastq samplereadstarget=10000}
\ControlFlowTok{done}
\end{Highlighting}
\end{Shaded}

Create output directory for this script (Change \texttt{irsa} with your \texttt{utenteX} name)

\begin{verbatim}
mkdir -p /home/irsa/analisi_16s/output_s02/
\end{verbatim}

Change its permission:
\texttt{chmod\ u+x\ s02\_subsampling.sh}

\section{Taxonomic Assignment}\label{taxonomic-assignment}

Last step, the sequences are compared to a reference database for taxonomic assignment and a relative abundance estimator for 16S genomic sequences.

\subsection{EMU}\label{emu}

Emu - species-level taxonomic abundance for full-length 16S reads.

This tool use a method optimized for error-prone full-length reads. However, it can be used for short-reads.

\url{https://github.com/treangenlab/emu}

To perform this annotation, we need a reference database that contains all the taxonomic information, which was already been downloaded in the following path: \texttt{/home/irsa/emu\_database}

\emph{Requirement}

To execute this tool, you need to activate the conda environment:

\texttt{mamba\ activate\ /home/irsa/miniconda3/envs/emu}

\textbf{STEP TIPS:}

\begin{itemize}
\tightlist
\item
  Your input data should be available from the previous step
\item
  Your task should be to use \texttt{emu\ abundance} on \textbf{each output files obtained from the previous step}, using the following options:

  \begin{itemize}
  \tightlist
  \item
    \texttt{-\/-type\ map-ont} denote sequencer {[}short-read:sr, Pac-Bio:map-pb, ONT:map-ont{]}
  \item
    \texttt{-\/-keep-counts} include estimated read counts for each species in output
  \item
    \texttt{-\/-output-dir\ \textless{}output\_dir\textgreater{}} directory for output results (to be created in advance)
  \item
    \texttt{-\/-output-basename\ \textless{}basename\_files\textgreater{}} basename of all output files saved in output-dir; default utilizes basename from input file(s)
  \end{itemize}
\item
  Create the folder for your output files (to use in your script)
\item
  To perform this step you should create a script (Recommended name: \texttt{s03\_abundance.sh}), and give it permission to be executed: \texttt{chmod\ u+x\ s03\_abundance.sh}

  \begin{itemize}
  \tightlist
  \item
    Specify the path of the EMU database with the following line as an action: \texttt{export\ EMU\_DATABASE\_DIR=/home/irsa/emu\_database}
  \item
    Last line of your script, should be an action that performs the command \texttt{emu\ combine-outputs} to create a single table containing all Emu output relative abundances in a single directory. Note this function will select all the .tsv files in the provided directory that contain `rel-abundance' in the filename. Use the following options in your \texttt{emu\ combine-outputs} command:

    \begin{itemize}
    \tightlist
    \item
      \texttt{-\/-counts\ output} estimated counts rather than relative abundance percentage in combined table. Only includes Emu relative abundance outputs that already have `estimated counts'
    \item
      \texttt{tax\_id} to get results for the most specific taxa level
    \end{itemize}
  \end{itemize}
\item
  Execute your script as follow: \texttt{./s03\_abundance.sh}
\item
  Look at the results
\end{itemize}

\textbf{{[}SPOILER{]}} - Scripts that we will use

We create an empty file called \texttt{s03\_abundance.sh}

\texttt{touch\ s03\_abundance.sh}

We can write our actions in the scripts as follows:

\begin{Shaded}
\begin{Highlighting}[]
\CommentTok{\#!/bin/bash}

\CommentTok{\# mamba activate /home/irsa/miniconda3/envs/emu}

\BuiltInTok{export} \VariableTok{EMU\_DATABASE\_DIR}\OperatorTok{=}\NormalTok{/home/irsa/emu\_database}

\ControlFlowTok{for}\NormalTok{ i }\KeywordTok{in}\NormalTok{ /home/irsa/analisi\_16s/output\_s02/}\PreprocessorTok{*}\NormalTok{{-}ss{-}nf.fastq}
\ControlFlowTok{do}
    \VariableTok{f}\OperatorTok{=}\VariableTok{$(}\FunctionTok{basename} \StringTok{"}\VariableTok{$i}\StringTok{"} \AttributeTok{{-}ss{-}nf.fastq}\VariableTok{)}
    \BuiltInTok{echo} \StringTok{"}\VariableTok{$f}\StringTok{"}
    \BuiltInTok{echo} \StringTok{"}\VariableTok{$i}\StringTok{"}
    \ExtensionTok{emu}\NormalTok{ abundance }\StringTok{"}\VariableTok{$i}\StringTok{"} \AttributeTok{{-}{-}type}\NormalTok{ map{-}ont }\AttributeTok{{-}{-}output{-}basename} \StringTok{"}\VariableTok{$f}\StringTok{"} \AttributeTok{{-}{-}keep{-}counts} \AttributeTok{{-}{-}output{-}dir}\NormalTok{ /home/irsa/analisi\_16s/output\_s03/ }\AttributeTok{{-}{-}threads}\NormalTok{ 4}
\ControlFlowTok{done}


\ExtensionTok{emu}\NormalTok{ combine{-}outputs }\AttributeTok{{-}{-}counts}\NormalTok{ /home/irsa/analisi\_16s/output\_s03/ tax\_id}

\end{Highlighting}
\end{Shaded}

Create output directory for this script (Change \texttt{irsa} with your \texttt{utenteX} name)

\begin{verbatim}
mkdir -p /home/irsa/analisi_16s/output_s03/
\end{verbatim}

Change its permission:
\texttt{chmod\ u+x\ s03\_abundance.sh}

\end{document}
